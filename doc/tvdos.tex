\chapter{Introduction}

\thedos\ is a Disk Operating System (usually) bundled with the distribution of the \thismachine.

All \thedos-related features requires the DOS to be fully loaded.




\chapter{Bootstrapping}

\index{boot process}\thedos\ goes through follwing progress to deliver the \code{A:/} prompt:

\section{Probing Bootable Devices}
BIOS

\section{The Bootloader}
LOADBOOT

Then the Bootsector will try to read and execute \code{A:/tvdos/TVDOS.SYS}

\section{TVDOS.SYS}
\thedos.SYS will load system libraries and variables and then will try to run the boot script by executing \code{A:\\AUTOEXEC.BAT}

\section{AUTOEXEC.BAT}

AUTOEXEC can setup user-specific variables (e.g. keyboard layout) and launch the command shell of your choice, \code{COMMAND} is the most common shell.

Variables can be set or changed using \textbf{SET} commands.



\chapter{DOS Commands}


\chapter{File I/O}
\index{filesystem (DOS)}In \thedos, drives are assigned with a drive letter, and the drive currently booted on is always drive \textbf{A}.


\section{The File Descriptor}
\index{file descriptor (DOS)}A file is virtualised through the \emph{file descriptor} which provides the functions to manipulate the file. Do note that when a file descriptor is created, the file is not yet opened by the drive.

To create a file descriptor, use the provided function \code{files.open(fullpath)}. \code{fullpath} is a fully qualified path of the file that includes the drive letter.

\section{Manipulating a File}
A file has folliwing properties and can be manipulated using following functions:

Properties:

\begin{outline}
\1\textbf{size}: Int
\\Returns a size of the file in bytes.
\1\textbf{path}: String
\\Returns a path (NOT including the drive letter) of the file. Paths are separated using reverse solidus.
\1\textbf{fullpath}: String
\\Returns a fully qualified path (including the drive letter) of the file. Paths are separated using reverse solidus.
\1\textbf{driverID}: String
\\Returns a filesystem driver ID associated with the file.
\1\textbf{driver}: [Object object]
\\Returns a filesystem driver (a Javascript object) for the file.
\1\textbf{isDirectory}: Boolean
\\Returns true if the path is a directory.
\1\textbf{name}: String
\\Returns the name part of the file's path.
\1\textbf{parentPath}: String
\\Returns a parent path of the file.
\1\textbf{exists}: Boolean
\\Returns true if the file exists on the device.
\end{outline}

Functions:

\begin{outline}
\1\textbf{pread}(pointer: Int, count: Int, offset: Int)
\\Reads the file bytewise and puts it to the memory starting from the \code{pointer}.
 \2\code{count}: how many bytes to read
 \2\code{offset}: when reading a file, how many bytes to skip initially
\1\textbf{bread}(): Array
\\Reads the file bytewise and returns the content in Javascript array.
\1\textbf{sread}(): String
\\Reads the file textwise and returns the content in Javascript string.
\1\textbf{pwrite}(pointer: Int, count: Int, offset: Int)
\\Writes the bytes stored in the memory starting from the \code{pointer} to file.
 \2\code{count}: how many bytes to write
 \2\code{offset}: when writing to the file, how many bytes on the file to skip before writing a first byte.
\1\textbf{bwrite}(bytes: UintArray)
\\Writes the bytes to the file.
\1\textbf{swrite}(string: String)
\\Writes the string to the file.
\1\textbf{flush}()
\\Flush the contents on the write buffer to the file immediately. Will do nothing if there is no write buffer implemented --- a write operation will always be performed imemdiately in such cases.
\1\textbf{close}()
\\Tells the underlying device (usually a disk drive) to close a file. When dealing with multiple files on a single disk drive (of which can only have a single active---or opened---file), the underlying filesystem driver will automatically swap the files around, so this function is normally unused.
\1\textbf{list}(): Array or undefined
\\Lists files inside of the directory. If the path is indeed a directory, an array of file descriptors will be returned; \code{undefined} otherwise.
\1\textbf{touch}(): Boolean
\\Updates the file's access time if the file exists; a new file will be created otherwise. Returns true if successful.
\1\textbf{mkDir}(): Boolean
\\Creates a directory to the path. Returns true if successful.
\1\textbf{mkFile}(): Boolean
\\Creates a new file to the path. Returns true if successful.
\1\textbf{remove}(): Boolean
\\Removes a file. Returns true if successful.
\end{outline}


\section{The Device Files}

\index{device file}Some devices are also virtualised through the file descriptor, and they are given a special path. (their fullpath does not contain a drive letter)

\begin{outline}
\1\textbf{RND} --- returns random bytes upon reading
 \2\textbf{pread}: returns the specified number of random bytes
\1\textbf{NUL} --- returns EOF upon reading
 \2\textbf{pread}: returns the specified number of EOFs
 \2\textbf{bread}: returns an empty array
 \2\textbf{sread}: returns an empty string
\1\textbf{ZERO} --- returns zero upon reading
 \2\textbf{pread}: returns the specified number of zeros
\1\textbf{CON} --- manipulates the screen text buffer, disregarding the colours
 \2\textbf{pread}: reads the texts as bytes.
 \2\textbf{bread}: reads the texts as bytes.
 \2\textbf{sread}: reads the texts as a string.
 \2\textbf{pwrite}: writes the bytes from the given pointer.
 \2\textbf{bwrite}: identical to \code{print()} except the given byte array will be casted to string.
 \2\textbf{swrite}: identical to \code{print()}.
\1\textbf{FBIPF} --- decodes IPF-formatted image to the framebuffer. Use the \emph{Graphics} library for the encoding.
 \2\textbf{pwrite, bwrite} --- decodes the given IPF binary data. \code{pwrite} offsets and counts are ignored.

\end{outline}


\chapter{DOS Libraries}


\section{The Input Library}

\dosnamespaceis{Input}{input}

\begin{outline}
\end{outline}



\section{The GL}

\dosnamespaceis{Graphics}{gl}

\begin{outline}
\end{outline}
