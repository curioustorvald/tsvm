\chapter{Introduction}

\thedos\ is a Disk Operating System (usually) bundled with the distribution of the \thismachine.

All \thedos-related features requires the DOS to be fully loaded.




\chapter{Bootstrapping}

\index{boot process}\thedos\ goes through follwing progress to deliver the \code{A:/} prompt:

\section{Probing Bootable Devices}
BIOS

\section{The Bootloader}
LOADBOOT

Then the Bootsector will try to read and execute \code{A:/tvdos/TVDOS.SYS}

\section{TVDOS.SYS}
\thedos.SYS will load system libraries and variables and then will try to run the boot script by executing \code{A:/AUTOEXEC.BAT}

\section{AUTOEXEC.BAT}

AUTOEXEC can setup user-specific variables (e.g. keyboard layout) and launch the command shell of your choice, \code{COMMAND} is the most common shell.

Variables can be set or changed using \textbf{SET} commands.



\chapter{DOS Commands}

\index{commands (DOS)}\index{coreutils (DOS)}DOS commands are only valid under the DOS environment.

To invoke the DOS commands from the Javascript-side, use:\\ \code{\_G.shell.coreutils.*}

\begin{outline}
\1\dossynopsis{cat}{file}{Reads a file and pipes its contents to the pipe, or to the console if no pipes are specified.}
\1\dossynopsis{cd}{dir}{Change the current working directory. Alias: chdir}
\1\dossynopsis{cls}{Clears the text buffer and the framebuffer if available.}
\1\dossynopsis{cp}{from to}{Make copies of the specified file. The source file must not be a directory. Alias: copy}
\1\dossynopsis{date}{Prints the system date. Alias: time}
\1\dossynopsis{dir}{path}{Lists the contents of the specifed path, or the current working directory if no arguments were given. Alias: ls}
\1\dossynopsis{del}{file}{Deletes the file. Aliases: erase, rm}
\1\dossynopsis{echo}{text}{Print the given text or a variable.}
\1\dossynopsis{exit}{Exits the current command processor.}
\1\dossynopsis{mkdir}{path}{Creates a directory. Aliase: md}
\1\dossynopsis{rem}{Comment-out the line.}
\1\dossynopsis{set}{key=value}{Sets the global variable \code{key} to \code{value}, or displays the list of global variables if no arguments were given.}
\1\dossynopsis{ver}{Prints the version of \thedos.}
\end{outline}



\chapter{File I/O}
\index{filesystem (DOS)}In \thedos, drives are assigned with a drive letter, and the drive currently booted on is always drive \textbf{A}.


\section{The File Descriptor}
\index{file descriptor (DOS)}A file is virtualised through the \emph{file descriptor} which provides the functions to manipulate the file. Do note that when a file descriptor is created, the file is not yet opened by the drive.

To create a file descriptor, use the provided function \code{files.open(fullpath)}. \code{fullpath} is a fully qualified path of the file that includes the drive letter.

\section{Manipulating a File}
A file has folliwing properties and can be manipulated using following functions:

Properties:

\begin{outline}
\1\propertysynopsis{size}{Int}{Returns a size of the file in bytes.}
\1\propertysynopsis{path}{String}{Returns a path (NOT including the drive letter) of the file. Paths are separated using reverse solidus.}
\1\propertysynopsis{fullpath}{String}{Returns a fully qualified path (including the drive letter) of the file. Paths are separated using reverse solidus.}
\1\propertysynopsis{driverID}{String}{Returns a filesystem driver ID associated with the file.}
\1\propertysynopsis{driver}{[Object object]}{Returns a filesystem driver (a Javascript object) for the file.}
\1\propertysynopsis{isDirectory}{Boolean}{Returns true if the path is a directory.}
\1\propertysynopsis{name}{String}{Returns the name part of the file's path.}
\1\propertysynopsis{parentPath}{String}{Returns a parent path of the file.}
\1\propertysynopsis{exists}{Boolean}{Returns true if the file exists on the device.}
\end{outline}

Functions:

\begin{outline}
\1\formalsynopsis{pread}{pointer: Int, count: Int, offset: Int}{Reads the file bytewise and puts it to the memory starting from the pointer.}
 \2\argsynopsis{count}{how many bytes to read}
 \2\argsynopsis{offset}{when reading a file, how many bytes to skip initially}
\1\formalsynopsis{bread}{}[Array]{Reads the file bytewise and returns the content in Javascript array.}
\1\formalsynopsis{sread}{}[String]{Reads the file textwise and returns the content in Javascript string.}
\1\formalsynopsis{pwrite}{pointer: Int, count: Int, offset: Int}
{Writes the bytes stored in the memory starting from the pointer to file.}
 \2\argsynopsis{count}{how many bytes to write}
 \2\argsynopsis{offset}{when writing to the file, how many bytes on the file to skip before writing a first byte.}
\1\formalsynopsis{bwrite}{bytes: UintArray}{Writes the bytes to the file.}
\1\formalsynopsis{swrite}{string: String}{Writes the string to the file.}
\1\formalsynopsis{flush}{}{Flush the contents on the write buffer to the file immediately. Will do nothing if there is no write buffer implemented --- a write operation will always be performed imemdiately in such cases.}
\1\formalsynopsis{close}{}{Tells the underlying device (usually a disk drive) to close a file. When dealing with multiple files on a single disk drive (of which can only have a single active---or opened---file), the underlying filesystem driver will automatically swap the files around, so this function is normally unused.}
\1\formalsynopsis{list}{}[Array or undefined]{Lists files inside of the directory. If the path is indeed a directory, an array of file descriptors will be returned; \code{undefined} otherwise.}
\1\formalsynopsis{touch}{}[Boolean]{Updates the file's access time if the file exists; a new file will be created otherwise. Returns true if successful.}
\1\formalsynopsis{mkDir}{}[Boolean]{Creates a directory to the path. Returns true if successful.}
\1\formalsynopsis{mkFile}{}[Boolean]{Creates a new file to the path. Returns true if successful.}
\1\formalsynopsis{remove}{}[Boolean]{Removes a file. Returns true if successful.}
\end{outline}


\section{The Device Files}

\index{device file}Some devices are also virtualised through the file descriptor, and they are given a special path. (their fullpath does not contain a drive letter)

\begin{outline}
\1\inlinesynopsis{RND}{returns random bytes upon reading}
 \2\argsynopsis{pread}{returns the specified number of random bytes}
\1\inlinesynopsis{NUL}{returns EOF upon reading}
 \2\argsynopsis{pread}{returns the specified number of EOFs}
 \2\argsynopsis{bread}{returns an empty array}
 \2\argsynopsis{sread}{returns an empty string}
\1\inlinesynopsis{ZERO}{returns zero upon reading}
 \2\argsynopsis{pread}{returns the specified number of zeros}
\1\inlinesynopsis{CON}{manipulates the screen text buffer, disregarding the colours}
 \2\argsynopsis{pread}{reads the texts as bytes.}
 \2\argsynopsis{bread}{reads the texts as bytes.}
 \2\argsynopsis{sread}{reads the texts as a string.}
 \2\argsynopsis{pwrite}{writes the bytes from the given pointer.}
 \2\argsynopsis{bwrite}{identical to \code{print()} except the given byte array will be casted to string.}
 \2\argsynopsis{swrite}{identical to \code{print()}.}
\1\inlinesynopsis{FBIPF}{decodes IPF-formatted image to the framebuffer. Use the \emph{Graphics} library for the encoding.}
 \2\argsynopsis{pwrite, bwrite}{decodes the given IPF binary data. Offsets and counts for \code{pwrite} are ignored.}

\end{outline}


\chapter{DOS Libraries}


\section{The Input Library}

\dosnamespaceis{Input}{input}

\begin{outline}
\end{outline}



\section{The GL}

\dosnamespaceis{Graphics}{gl}

\begin{outline}
\end{outline}
