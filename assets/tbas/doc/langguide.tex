\quad
\chapterprecishere{``Begin at the beginning'', the King said gravely, ``and go on till you come to the end: then stop.''\par\raggedleft --- \textup{Lewis Carroll, } Alice in Wonderland}

We'll begin at the beginning; how beginning? This:

\begin{lstlisting}
10 PRINT 2+2
run
4
Ok
\end{lstlisting}

Oh \emph{boy} we just did a computation! It printed out \code{4} which is a correct answer for $2+2$ and it didn't crash!

\begin{lstlisting}
10 DEFUN FAC(N)=IF N==0 THEN 1 ELSE N*FAC(N-1)
20 FOR K=1 TO 6
30 PRINT FAC(K)
40 NEXT
\end{lstlisting}

\begin{lstlisting}
10 DEFUN FAC(N)=IF N==0 THEN 1 ELSE N*FAC(N-1)
20 K=MAP FAC, 1 TO 10
30 PRINT K
\end{lstlisting}


\begin{lstlisting}
10 DEFUN FIB(N)=IF N==0 THEN 0 ELSE IF N==1 THEN 1 ELSE FIB(N-1)+FIB(N-2)
20 FOR K=1 TO 12
30 PRINT FIB(K);" ";
40 NEXT
\end{lstlisting}

\section[Currying]{Haskell Curry Wants to Know Your Location}

So what the fsck is currying? Consider the following code:

\begin{lstlisting}
10 DEFUN F(K,T)=ABS(T)==K
20 CF=CURRY(F,32)
30 PRINT CF(24) : REM will print 'false'
40 PRINT CF(-32) : REM will print 'true'
\end{lstlisting}

Here, \code{CF} is a curried function of \code{F}; built-in function \code{CURRY} applies \code{32} to the first parametre of the function \code{F}, which dynamically returns a \emph{function} of \code{CF(T) = ABS(T) == 32}. The fact that \code{CURRY} returns a \emph{function} opens many possibilities, for example, you can create loads of sibling functions without making loads of duplicate codes.

\section[Wrapping-Up]{The Grand Unification}

Using all the knowledge we have learned, it should be trivial\footnote{/s} to write a Quicksort function in \tbas, like this:

\begin{lstlisting}
10 DEFUN LESS(P,X)=X<P
11 DEFUN GTEQ(P,X)=X>=P
12 DEFUN QSORT(XS)=IF LEN(XS)<1 THEN NIL ELSE QSORT(FILTER(CURRY(LESS,XS(0)),XS)) # XS(0)!NIL # QSORT(FILTER(CURRY(GTEQ,XS(0)),XS))
100 L=7!9!4!5!2!3!1!8!6!NIL
110 SL=QSORT(L)
120 PRINT L:PRINT SL
\end{lstlisting}

Line 12 implements quicksort algorithm, using \code{LESS} and \code{GTEQ} as helper functions. \code{LESS} is a user-function version of less-than operator, and \code{GTEQ} is similar. \code{QSORT} selects a pivot using head-element of array \code{XS}\footnote{stands for \emph{X's}}, then using curried version of \code{LESS} and \code{GTEQ} to move values lesser than pivot to the left and greater to the right, and these two sorted \emph{chunks} are recursively sorted using the same \code{QSORT} function. Currying is used to give comparison functions a pivot-value to compare against, and also because \code{FILTER} wants a \emph{function} and not an \emph{expression}. \code{XS(0)} simply fetches a head-elemement of \code{XS}. \code{XS(0)!NIL} creates a single-element array contains \code{XS(0)}.

%Uncomment this if you finally decided to support a closure%
%% Using \emph{closure}, the definition of \code{QSORT} can be \emph{even more cryptic}:
%% 
%% \begin{lstlisting}
%% 10 QSORT=[XS]~>IF LEN(XS)<1 THEN NIL ELSE QSORT(FILTER([K]~>K<XS(0),XS)) # XS(0)!NIL # QSORT(FILTER([K]~>K>=XS(0),XS))
%% 100 L=7!9!4!5!2!3!1!8!6!NIL
%% 110 SL=QSORT(L)
%% 120 PRINT L:PRINT SL
%% \end{lstlisting}
